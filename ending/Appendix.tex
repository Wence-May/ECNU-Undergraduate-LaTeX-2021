\section*{\centerline{\bfseries \sffamily \zihao{-3} 附录}}

\subsection{部分代码}
\begin{itemize}
\item 定义PercepterBase类做一些简单的工作如logger、场景判断。
\end{itemize}
\begin{lstlisting}
export default abstract class PerceptorBase {
    public logger: any;

    constructor() {
    this.logger = {
        info: (message?: any, ...optionalParams: any[]): 
        void => {
        if (process.env.NODE_ENV !== 'production') {
            console.log('ℹ️ Perceptor : ',
             message, ...optionalParams);
        }
        },
        error: (message?: any, ...optionalParams: any[]): 
        void => {
        console.error('❌ Error Message From Perceptor :',
         message, ...optionalParams);
        // TODO: pass error message to ErrorMessageBar component
        render(<ErrorMessageBar />, 
        document.getElementById('perceptor'))
        }
    };
    }

    public abstract run(): Promise<void>;
}
\end{lstlisting}

\begin{itemize}
\item 定义Perceptor类继承PerceptorBase,来检查settings、创建div、取配置文件、渲染每一个组件。
\end{itemize}
\begin{lstlisting}
export class Perceptor extends PerceptorBase {
    public static Features: Map<string, any> = new Map();
    public settings: any;

    public async run(): Promise<void> {
    // wait until <body> element is ready
    await elementReady('body', { waitForChildren: false });
    this.logger.info('body element is ready.');

    this.logger.info('creating perceptor div ...');
    const perceptorDiv = document.createElement('div');
    perceptorDiv.id = 'perceptor';
    $('#js-repo-pjax-container').prepend(perceptorDiv);

    await this.checkSettings();

    // run every features
    Perceptor.Features.forEach(async (Feature, name) => {
        const featureId = name.replace(name[0],
        name[0].toLowerCase());
        this.logger.info('trying to load ', featureId)
        if (this.settings.toJson()[featureId] === false) {
        this.logger.info(featureId, 'is disabled');
        return;
        }
        if (Feature.prototype.include
        .every((c: () => any) => !c())) {
        this.logger.info(featureId,
         'does NOT run on this page')
        return;
        }
        try {
        this.logger.info('running ', featureId)
        const feature = new Feature();
        await feature.run();
        } catch (error: unknown) {
        this.logger.error(featureId, error)
        }
    }, this)

    private async checkSettings(): Promise<void> {
        this.logger.info('loading settings ...');
        if (isRepo()) {
            this.logger.info('Detected that this is a repo page,
             trying to load configuration file from the repo ...');

            const owner = utils.getRepositoryInfo(window.location)!
            .owner;
            const repo = utils.getRepositoryInfo(window.location)!
            .name;
            const configFromGithub = await getConfigFromGithub(owner, repo);
            this.logger.info('The configurations are: ', configFromGithub);
            this.settings = await mergeSettings(configFromGithub);
        } else {
            this.settings = await loadSettings();
        }
    }
}
\end{lstlisting}
\begin{itemize}
    \item 渲染组件,以开发者协作网络为例:
\end{itemize}
\begin{lstlisting}
public async run(): Promise<void> {
    const pinnedReposDiv = $('.js-pinned-items-reorder-container')
    .parent();
    const DeveloperNetworkDiv = document.createElement('div');
    DeveloperNetworkDiv.id = 'developer-network';
    DeveloperNetworkDiv.style.width = "100%";
    this._currentDeveloper = $('.p-nickname.vcard-username.d-block')
    .text().trim();
    const settings=await loadSettings();
    try {
    const forceGraphDataRaw = await getGraphData(`/actor/${this._currentDeveloper}.json`);
    await this.generateForceGraphData(forceGraphDataRaw);

    const circularGraphDataRaw = await getGraphData(
        `/actor/${this._currentDeveloper}_top.json`);
    await this.generateCircularGraphData(circularGraphDataRaw);

    const developerColumns = [
        {
        key: 'column1',
        name: getMessageI18n('global_developer'),
        fieldName: 'name',
        minWidth: 100,
        maxWidth: 200,
        isResizable: true,
        onRender: (item: any) => (
            <Link href={'https://github.com/' + item.name} >
            {item.name}
            </Link>
        ),
        },
        { key: 'column2', name: getMessageI18n('global_correlation'), fieldName: 'correlation', minWidth: 100, maxWidth: 200, isResizable: true },
        { key: 'column3', name: getMessageI18n('global_activity'), fieldName: 'activity', minWidth: 100, maxWidth: 200, isResizable: true },
    ];
    const repoColumns = [
        {
        key: 'column1',
        name: getMessageI18n('global_repo'),
        fieldName: 'name',
        minWidth: 100,
        maxWidth: 200,
        isResizable: true,
        onRender: (item: any) => (
            <Link href={'https://github.com/' + item.name} >
            {item.name}
            </Link>
        ),
        },
        { key: 'column2', name:
         getMessageI18n('global_contribution'), fieldName: 'value', minWidth: 100, maxWidth: 200, isResizable: true },
    ];
    render(
        <div>
        < GraphWithList
            layout='force'
            graphType={settings.graphType}
            title={getMessageI18n('component_developerCollabrationNetwork_title')}
            graphData={this._forceGraphData}
            graphDataGraphin={this._forceGraphDataGraphin}
            columns={developerColumns}
            listData={this._developerListData}
        />
        < GraphWithList
            layout='circular'
            graphType={settings.graphType}
            title={getMessageI18n('component_mostParticipatedProjects_title')}
            graphData={this._circularGraphData}
            graphDataGraphin={this._circularGraphDataGraphin}
            columns={repoColumns}
            listData={this._repoListData}
        />
        </div>,
        DeveloperNetworkDiv,
    );
    pinnedReposDiv.before(DeveloperNetworkDiv);
    } catch (error) {
    this.logger.error('DeveloperNetwork', error);
    return;
    }
}
\end{lstlisting}

