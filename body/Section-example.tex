%  Section A
\section{章节结构测试}这节用来展示文章的5层结构。事实上,一般来说文章层次在3-4层为宜。在之后的section中,我们会只使用至多3层结构(即,节-小节-子节)来进行各种演示。
 
\subsection{小节标题}这一小节我们介绍这些内容。

\subsubsection{子节标题}这一子节我们介绍这些内容。

\paragraph{段标题}这一段我们介绍这些内容。 

\subparagraph{小段标题}这一小段我们介绍这些内容。

%   Section B
\section{定理等环境测试}这节用来展示定理,引理等常用论文环境。
 
\subsection{编号环境与不编号环境}

\subsubsection{编号环境}

\begin{theorem}\label{thm2_1}

    设$A,B$是两个实数, 则$2AB\leq 2 A^2+B^2$.
    
\end{theorem}

\begin{proof}
    这里是证明。
\end{proof}

\begin{lemma}[Nakayama引理]\label{lem2_2}
    这是一条引理测试。。。
\end{lemma}

\begin{problem}[连续统假设]是否存在$\mathbb{R}$的子集S使得$card(\mathbb{N})<card(S)<card(\mathbb{R})$?
\end{problem}
\begin{solution}
    不存在。
\end{solution}

\subsubsection{无编号环境}

\begin{theorem*}

    设$A,B$是两个实数, 则$2AB\leq 2 A^2+B^2$.
    
\end{theorem*}

\begin{proof}
    这里是证明。
\end{proof}

\begin{lemma*}[Nakayama引理]
    这是一条引理测试。。。
\end{lemma*}

\begin{problem*}[连续统假设]是否存在$\mathbb{R}$的子集S使得$card(\mathbb{N})<card(S)<card(\mathbb{R})$?
\end{problem*}
\begin{solution}
    不存在。
\end{solution}

% Section C
\section{公式测试}这节用来展示公式,交换图等。
 
\subsection{行内公式}
典范的同态$\lim_{\leftarrow F} W_r(S)\rightarrow \lim_{\leftarrow F} W_r(S/\pi S )$是同构。

\subsection{整行公式}
$$\mathbb{A}_{inf}=W(S^\flat)\cong \lim_{\leftarrow F} W_r(S)$$

\subsection{多行公式}
\begin{sloppypar}
多行公式的情况非常多,对齐与换行的要求也各不相同。所以选择合适的环境非常重要。这份文档里无法涵盖所有情况,所以提供一个教程用以参考:\url{http://blog.csdn.net/yanxiangtianji/article/details/54767265}
\end{sloppypar}



\subsubsection{align环境}
\begin{align*}
    \operatorname{E} (Z_{n+1} - Z_n | X_1,..., X_n)
    &= \operatorname{E} (S_{n+1}^2 - (n+1) \sigma^2 - S_n^2 + n \sigma^2 | X_1,..., X_n) \\
    &= \operatorname{E} (S_{n+1}^2 - S_n^2 - (n+1) \sigma^2 + n \sigma^2 | X_1,..., X_n) \\
    &= \operatorname{E} (X_{n+1}(X_{n+1} + 2\sum_{i=1}^n X_i) - \sigma^2 | X_1,..., X_n) \\
    &= \operatorname{E} (X_{n+1}X_{n+1})
       + 2\operatorname{E} (X_{n+1}) \sum_{i=1}^n X_i - \sigma^2 \\
    &= \sigma^2  - \sigma^2 =0.
\end{align*}

\subsubsection{split环境(内嵌)}
\begin{equation*}
    \begin{split}
    (a + b)^4
      &= (a + b)^2 (a + b)^2      \\
      &= (a^2 + 2ab + b^2)
         (a^2 + 2ab + b^2)        \\
      &= a^4 + 4a^3b + 6a^2b^2 + 4ab^3 + b^4
    \end{split}
\end{equation*}

\subsubsection{带大括号的多行公式}
\paragraph{cases}
$$
    f=
    \begin{cases}
      x + y = z,  \\
      1 + 2 = 3.  \\
    \end{cases}
$$

\paragraph{array}
$$ F^{HLLC}=\left\{
\begin{array}{rcl}
F_L       &      & {0      <      S_L}\\
F^*_L     &      & {S_L \leq 0 < S_M}\\
F^*_R     &      & {S_M \leq 0 < S_R}\\
F_R       &      & {S_R \leq 0}
\end{array} \right. $$
    
\paragraph{aligned}
\begin{equation}
    \left\{
     \begin{aligned}
     \overset{.}x(t) &=A_{ci}x(t)+B_{1ci}w(t)+B_{2ci}u(t)  \\
     z(t) &=C_{ci}x(t)+D_{ci}u(t) \\
     \end{aligned}
     \right.
\end{equation}

\subsection{交换图}
\begin{sloppypar}
强烈推荐tikzcd-editor:\url{https://github.com/yishn/tikzcd-editor}
\end{sloppypar}

\begin{center}
\begin{tikzcd}
    T
    \arrow[drr, bend left, "x"]
    \arrow[ddr, bend right, "y"]
    \arrow[dr, dotted, "{(x,y)}" description] & & \\
    & X \times_Z Y \arrow[r, "p"] \arrow[d, "q"]
    & X \arrow[d, "f"] \\
    & Y \arrow[r, "g"]
    & Z
\end{tikzcd}
\end{center}

\begin{center}
    \begin{tikzcd}[row sep=scriptsize, column sep=scriptsize]
        & f^* E_V \arrow[dl] \arrow[rr] \arrow[dd] & & E_V \arrow[dl] \arrow[dd] \\
        f^* E \arrow[rr, crossing over] \arrow[dd] & & E \\
        & U \arrow[dl] \arrow[rr] & & V \arrow[dl] \\
        M \arrow[rr] & & N \arrow[from=uu, crossing over]\\
        \end{tikzcd}
\end{center}

\begin{center}
\begin{tikzpicture}[commutative diagrams/every diagram]
    \node (P0) at (90:2.3cm) {$X\otimes (Y\otimes (Z\otimes T))$};
    \node (P1) at (90+72:2cm) {$X\otimes ((Y\otimes Z)\otimes T))$} ;
    \node (P2) at (90+2*72:2cm) {\makebox[5ex][r]{$(X\otimes (Y\otimes Z))\otimes T$}};
    \node (P3) at (90+3*72:2cm) {\makebox[5ex][l]{$((X\otimes Y)\otimes Z)\otimes T$}};
    \node (P4) at (90+4*72:2cm) {$(X\otimes Y)\otimes (Z\otimes T)$};
    \path[commutative diagrams/.cd, every arrow, every label]
    (P0) edge node[swap] {$1\otimes\phi$} (P1)
    (P1) edge node[swap] {$\phi$} (P2)
    (P2) edge node {$\phi\otimes 1$} (P3)
    (P4) edge node {$\phi$} (P3)
    (P0) edge node {$\phi$} (P4);
\end{tikzpicture}
\end{center}


% Section D
\subsection{表格}
\par 本来LaTeX里表格的变化是非常多的,但鉴于学校要求用三线式,问题反而简单了。以下是一个例子:
\begin{table}[htbp]\center
    \caption{示例表格\\Example Table}
    \begin{tabular}{lcccccl}
     \toprule
     。。 & 。。 & 。。 & 。。 & 。。& 。。 & 。。\\
     \midrule
    。。 & 。。 & 。。 & 。。 & 。。& 。。 & 。。\\
    。。 & 。。 & 。。 & 。。 & 。。& 。。 & 。。\\
    。。 & 。。 & 。。 & 。。 & 。。& 。。 & 。。\\
    。。 & 。。 & 。。 & 。。 & 。。& 。。 & 。。\\
    。。 & 。。 & 。。 & 。。 & 。。& 。。 & 。。\\
     \bottomrule
    \end{tabular}
   \end{table}
如果你有使用更复杂的表格的需求,请自行查资料完成。

\subsection{插图}
由于这份模板不考虑多栏排版,所以格式要求中所述的半栏图大小要求我们不作演示。以下是一个通栏图的演示:
\begin{figure}[H]
    \centering
    \includegraphics[width=100mm]{example-image}
    \caption{图片测试(最小宽度)\\Image test (Minimal width)}
  \end{figure}

\begin{figure}[H]
    \centering
    \includegraphics[width=130mm]{example-image}
    %\includegraphics[width=130mm]{./figures/你自己的图像文件}
    \caption{图片测试(最大宽度)\\Image test (Maximal width)}
\end{figure}
\par 注意:这里为了减少图片上下的空白,使用了float宏包。
© 2021 GitHub, Inc.


% Section E
\section{注释与引用}这节用来展示注释与引用。

\subsection{注释——脚注与尾注}
\subsubsection{脚注}
\par 这里是脚注测试\footnote{1111111111}这里是脚注测试这里是脚注测试这里是脚注测试\footnote{2222222222}这里是脚注测试这里是脚注测试这里是脚注测试这里是脚注测试这里是脚注测试这里是脚注测试这里是脚注测试这里是脚注测试这里是脚注测试这里是脚注测试这里是脚注测试这里是脚注测试这里是脚注测试这里是脚注测试这里是脚注测试\footnote{3333333333}这里是脚注测试这里是脚注测试这里是脚注测试这里是脚注测试这里是脚注测试这里是脚注测试这里是脚注测试这里是脚注测试这里是脚注测试这里是脚注测试这里是脚注测试这里是脚注测试

\textcolor{red}{\textbf{\uline{注意!正如这份演示中所出现的情况,若该页(也就是本文档中的前一页)剩余空间不大,不足以显示足够多的文档与脚注,那么该段文字就会被移至下一页而留下空白。目前我们尚未找到解决的方法,所以如果遇到了这个问题,请修改排版,以留下足够大的空间。}}}

\subsubsection{尾注}
\par 这里是尾注测试\endnote{伴随着互联网的发展以及新的网络应用的出现,互联网用户由单纯的“读”网页,向“读、写”网页,共同建设互联网发展,由此网上产生了大量带有用户主观感情的数据,从这些带...}这里是尾注测试这里是尾注测试这里是尾注测试这里是尾注测试\endnote{尾注测试2}这里是尾注测试这里是尾注测试这里是尾注测试这里是尾注测试这里是尾注测试这里是尾注测试这里是尾注测试这里是尾注测试这里是尾注测试这里是尾注测试这里是尾注测试这里是尾注测试这里是尾注测试这里是尾注测试这里是尾注测试\endnote{尾注测试3}这里是尾注测试这里是尾注测试这里是尾注测试这里是尾注测试这里是尾注测试这里是尾注测试这里是尾注测试这里是尾注测试这里是尾注测试

\par \textcolor{red}{\textbf{\uline{注意!endnotes宏包并不支持hyperref,也就是无法通过点击文中尾注标号以跳转到尾注。当然,这在打印出来的文档中并不会造成任何影响。}}}
\par \textcolor{blue}{\textbf{\uline{提示:尾注出现在全文最后。为了区分脚注与尾注的编号,我们在尾注编号前加上了“尾注”二字。}}}

\subsection{交叉引用}
\par 本模板使用cleveref宏包来进行交叉引用。使用的指令为$\backslash$cref$\{$label$\}$。例子如下:
\par 由\cref{thm2_1}我们可以知道XXXXXXXX。
\par 由\cref{lem2_2}我们可以知道XXXXXXXX。
\par 请注意,label是需要手工设置的,一般将label放在你需要引用的环境内即可(具体可见SectionB.tex)。

\subsection{文献引用的演示}
\par 本模板使用biblatex进行文献管理,这是一套相对较新的系统。另外,使用了hushidong制作的符合gb7714-2015标准的biblatex样式。在此对他的工作表示感谢,要完成这样的样式非常不容易。本模板中gb7714-2015.bbx与gb7714-2015.cbx即为他的作品,在这里打包发布以便使用。
\par 默认的bib文件位于~/reference/thesis-ref.bib,内容是由Wang Tianshu制作,在此仅作演示之用。关于bib文件的编写与管理请自行查找相关教程。
\par 下方的演示已经给出了正文中引用文献的基本方法,这与传统的cite命令是类似的。如有更多需求,请至\url{https://github.com/hushidong/biblatex-gb7714-2015}查找相关资料。
\par 文献\parencite{Yang_Hy200215}中提到xxxxxxx。
\par 文献\parencite{Joa1999}中提到yyyyyyy。
\par 文献\parencite{Altman1997}中提到zzzzzzz。
\par \textcolor{blue}{\textbf{\uline{本模板使用parencite而不是cite命令,因为这样能与脚注所产生编号进行区分。当然,如果你没有脚注或尾注,那么cite命令也是推荐使用的。}}}