\section{总结与展望}
% 项目本身的意义、对开源社区做出的贡献
% 评价指标是什么(?)
% //TODO:不知道如何给出客观评价指标
\subsection{总结}
\par 本文围绕着开放平台的数据可视化问题,对迷你看板——一个浏览器插件落地所经历的的分析、设计与实现过程进行了详细的阐述。
\par 首先阐述了项目的研究背景与研究动机,主要介绍了开源社区与开放组织的的运作模式与发展方向,随之提出了本文主要面向的问题与需求。接着,带着问题与需求,去对开放平台的数据进行探索分析,挖掘其隐藏并有价值的数据信息,并形成具体的的可视化分析的方向。明确了研究方向与目的以后,再从产品与业务的角度出发,剖析需求、流程、业务逻辑,完成迷你看板的总体定位与细节功能设计。最后,针对每一个开发环节与功能模块,调研技术方法、确定技术方法,并逐个实现。

\subsection{展望}
\par 经由这个项目,在开源社区中可视化深层信息、辅助决策的目的已经基本能够达到。这个迷你看板以浏览器插件的形式实现,在完善开源社区机制、推动生产力关系变革的进程中迈进了有意义的一步,在这个项目的基础上,已然可以展望开源供应链的迭代、区块链等新技术的加入,但这些也绝不会是终点。开源的进程在当前的基础上将不断被推动,并将最终演化出新的组织关系和技术产物,相互促进,往复循环。